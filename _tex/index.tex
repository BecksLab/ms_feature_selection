% Options for packages loaded elsewhere
% Options for packages loaded elsewhere
\PassOptionsToPackage{unicode}{hyperref}
\PassOptionsToPackage{hyphens}{url}
\PassOptionsToPackage{dvipsnames,svgnames,x11names}{xcolor}
%
\documentclass[
]{article}
\usepackage{xcolor}
\usepackage{amsmath,amssymb}
\setcounter{secnumdepth}{5}
\usepackage{iftex}
\ifPDFTeX
  \usepackage[T1]{fontenc}
  \usepackage[utf8]{inputenc}
  \usepackage{textcomp} % provide euro and other symbols
\else % if luatex or xetex
  \usepackage{unicode-math} % this also loads fontspec
  \defaultfontfeatures{Scale=MatchLowercase}
  \defaultfontfeatures[\rmfamily]{Ligatures=TeX,Scale=1}
\fi
\usepackage{lmodern}
\ifPDFTeX\else
  % xetex/luatex font selection
\fi
% Use upquote if available, for straight quotes in verbatim environments
\IfFileExists{upquote.sty}{\usepackage{upquote}}{}
\IfFileExists{microtype.sty}{% use microtype if available
  \usepackage[]{microtype}
  \UseMicrotypeSet[protrusion]{basicmath} % disable protrusion for tt fonts
}{}
\makeatletter
\@ifundefined{KOMAClassName}{% if non-KOMA class
  \IfFileExists{parskip.sty}{%
    \usepackage{parskip}
  }{% else
    \setlength{\parindent}{0pt}
    \setlength{\parskip}{6pt plus 2pt minus 1pt}}
}{% if KOMA class
  \KOMAoptions{parskip=half}}
\makeatother
% Make \paragraph and \subparagraph free-standing
\makeatletter
\ifx\paragraph\undefined\else
  \let\oldparagraph\paragraph
  \renewcommand{\paragraph}{
    \@ifstar
      \xxxParagraphStar
      \xxxParagraphNoStar
  }
  \newcommand{\xxxParagraphStar}[1]{\oldparagraph*{#1}\mbox{}}
  \newcommand{\xxxParagraphNoStar}[1]{\oldparagraph{#1}\mbox{}}
\fi
\ifx\subparagraph\undefined\else
  \let\oldsubparagraph\subparagraph
  \renewcommand{\subparagraph}{
    \@ifstar
      \xxxSubParagraphStar
      \xxxSubParagraphNoStar
  }
  \newcommand{\xxxSubParagraphStar}[1]{\oldsubparagraph*{#1}\mbox{}}
  \newcommand{\xxxSubParagraphNoStar}[1]{\oldsubparagraph{#1}\mbox{}}
\fi
\makeatother


\usepackage{longtable,booktabs,array}
\newcounter{none} % for unnumbered tables
\usepackage{calc} % for calculating minipage widths
% Correct order of tables after \paragraph or \subparagraph
\usepackage{etoolbox}
\makeatletter
\patchcmd\longtable{\par}{\if@noskipsec\mbox{}\fi\par}{}{}
\makeatother
% Allow footnotes in longtable head/foot
\IfFileExists{footnotehyper.sty}{\usepackage{footnotehyper}}{\usepackage{footnote}}
\makesavenoteenv{longtable}
\usepackage{graphicx}
\makeatletter
\newsavebox\pandoc@box
\newcommand*\pandocbounded[1]{% scales image to fit in text height/width
  \sbox\pandoc@box{#1}%
  \Gscale@div\@tempa{\textheight}{\dimexpr\ht\pandoc@box+\dp\pandoc@box\relax}%
  \Gscale@div\@tempb{\linewidth}{\wd\pandoc@box}%
  \ifdim\@tempb\p@<\@tempa\p@\let\@tempa\@tempb\fi% select the smaller of both
  \ifdim\@tempa\p@<\p@\scalebox{\@tempa}{\usebox\pandoc@box}%
  \else\usebox{\pandoc@box}%
  \fi%
}
% Set default figure placement to htbp
\def\fps@figure{htbp}
\makeatother


% definitions for citeproc citations
\NewDocumentCommand\citeproctext{}{}
\NewDocumentCommand\citeproc{mm}{%
  \begingroup\def\citeproctext{#2}\cite{#1}\endgroup}
\makeatletter
 % allow citations to break across lines
 \let\@cite@ofmt\@firstofone
 % avoid brackets around text for \cite:
 \def\@biblabel#1{}
 \def\@cite#1#2{{#1\if@tempswa , #2\fi}}
\makeatother
\newlength{\cslhangindent}
\setlength{\cslhangindent}{1.5em}
\newlength{\csllabelwidth}
\setlength{\csllabelwidth}{3em}
\newenvironment{CSLReferences}[2] % #1 hanging-indent, #2 entry-spacing
 {\begin{list}{}{%
  \setlength{\itemindent}{0pt}
  \setlength{\leftmargin}{0pt}
  \setlength{\parsep}{0pt}
  % turn on hanging indent if param 1 is 1
  \ifodd #1
   \setlength{\leftmargin}{\cslhangindent}
   \setlength{\itemindent}{-1\cslhangindent}
  \fi
  % set entry spacing
  \setlength{\itemsep}{#2\baselineskip}}}
 {\end{list}}
\usepackage{calc}
\newcommand{\CSLBlock}[1]{\hfill\break\parbox[t]{\linewidth}{\strut\ignorespaces#1\strut}}
\newcommand{\CSLLeftMargin}[1]{\parbox[t]{\csllabelwidth}{\strut#1\strut}}
\newcommand{\CSLRightInline}[1]{\parbox[t]{\linewidth - \csllabelwidth}{\strut#1\strut}}
\newcommand{\CSLIndent}[1]{\hspace{\cslhangindent}#1}



\setlength{\emergencystretch}{3em} % prevent overfull lines

\providecommand{\tightlist}{%
  \setlength{\itemsep}{0pt}\setlength{\parskip}{0pt}}



 


\makeatletter
\@ifpackageloaded{caption}{}{\usepackage{caption}}
\AtBeginDocument{%
\ifdefined\contentsname
  \renewcommand*\contentsname{Table of contents}
\else
  \newcommand\contentsname{Table of contents}
\fi
\ifdefined\listfigurename
  \renewcommand*\listfigurename{List of Figures}
\else
  \newcommand\listfigurename{List of Figures}
\fi
\ifdefined\listtablename
  \renewcommand*\listtablename{List of Tables}
\else
  \newcommand\listtablename{List of Tables}
\fi
\ifdefined\figurename
  \renewcommand*\figurename{Figure}
\else
  \newcommand\figurename{Figure}
\fi
\ifdefined\tablename
  \renewcommand*\tablename{Table}
\else
  \newcommand\tablename{Table}
\fi
}
\@ifpackageloaded{float}{}{\usepackage{float}}
\floatstyle{ruled}
\@ifundefined{c@chapter}{\newfloat{codelisting}{h}{lop}}{\newfloat{codelisting}{h}{lop}[chapter]}
\floatname{codelisting}{Listing}
\newcommand*\listoflistings{\listof{codelisting}{List of Listings}}
\makeatother
\makeatletter
\makeatother
\makeatletter
\@ifpackageloaded{caption}{}{\usepackage{caption}}
\@ifpackageloaded{subcaption}{}{\usepackage{subcaption}}
\makeatother
\usepackage{bookmark}
\IfFileExists{xurl.sty}{\usepackage{xurl}}{} % add URL line breaks if available
\urlstyle{same}
\hypersetup{
  pdftitle={Multiple dimensions of network structure underlie ecological stability},
  pdfauthor={Tanya Strydom; Andrew P. Beckerman},
  pdfkeywords={food web, structure, dimensionality reduction},
  colorlinks=true,
  linkcolor={blue},
  filecolor={Maroon},
  citecolor={Blue},
  urlcolor={Blue},
  pdfcreator={LaTeX via pandoc}}



\title{Multiple dimensions of network structure underlie ecological
stability}
\author{Tanya Strydom %
%
\textsuperscript{%
%
1%
}%
; Andrew P. Beckerman %
%
\textsuperscript{%
%
1%
}%
}
\date{2026-02-24}

\usepackage{setspace}
\usepackage[left]{lineno}
\usepackage[letterpaper]{geometry}

\usepackage[nolists,noheads,markers]{endfloat}
\geometry{margin=2.5cm}

\begin{document}

\thispagestyle{empty}
{\bfseries\sffamily\Large Multiple dimensions of network structure
underlie ecological stability}
\vfil
Tanya Strydom %
%
\textsuperscript{%
%
1%
}%
; Andrew P. Beckerman %
%
\textsuperscript{%
%
1%
}%

\vfil
{\small
\textbf{Abstract:} Ecological network structure is widely invoked to
explain patterns of biodiversity and stability, yet decades of research
have produced little consensus on which structural descriptors matter
most. We argue that this apparent inconsistency arises not from
methodological disagreement, but from a conceptual mismatch: network
structure encodes multiple, hierarchically organised aspects of energy
flow, each relating to different components of stability. Using a
compilation of food webs from XXX, we quantified 31 commonly used
structural metrics spanning species roles, interaction pathways, network
geometry, and emergent system behaviour. We first used variable
clustering and dimensionality reduction to identify dominant axes of
structural variation and a non-redundant subset of descriptors. We then
interpreted these axes through an energy-flow framework linking
structural scale to distinct stability mechanisms: persistence
(maintenance of energy flow following species loss), resistance
(attenuation of perturbation propagation), and return (reorganisation of
pathways following disturbance). Our analyses show that metrics often
treated as competing predictors of `stability' instead load onto
different structural dimensions and relate to different stability
components. Redundancy and species-role metrics primarily align with
persistence, pathway and motif-based metrics with resistance, and global
organisational measures with return dynamics. By explicitly mapping
network structure to stability mechanisms, our results reconcile
conflicting findings in the structure--stability literature and provide
a principled framework for selecting network metrics based on ecological
question and scale, rather than convention.
\vfil
\textbf{Keywords:} %
food web, structure, %
dimensionality reduction%
}
\clearpage
\setcounter{page}{1}
\doublespacing
\linenumbers


\section{Introduction}\label{introduction}

\begin{quote}
Which metrics best represent the distinct stages of the energy-flow
hierarchy, and do they capture different components of stability?
\end{quote}

Ecological networks are commonly characterised using a large and diverse
set of structural metrics, yet there remains little consensus on how
these metrics should be interpreted or compared across studies (Vermaat
et al.~2009; Lau et al.~2017). We argue that this difficulty arises
because network structure is not a single property, but a hierarchy of
interrelated components that encode different aspects of how energy
moves through an ecosystem (Ulanowicz 1986; Thompson et al.~2012).
Metrics derived at different structural scales therefore capture
different ecological processes and, consequently, relate to different
components of stability. Making these distinctions explicit is essential
for interpreting patterns of structure--stability relationships.

\subsubsection{Structural scale and the organisation of energy
flow}\label{structural-scale-and-the-organisation-of-energy-flow}

We conceptualise food-web structure as organised across four
hierarchical structural scales: node-level, path-level, network
geometry, and emergent system behaviour (Pimm 1982; Cohen et al.~1990;
Vermaat et al.~2009). These scales reflect increasing levels of
integration, progressing from individual species roles to whole-network
dynamics.

\textbf{Node-level structure:} describes the properties and roles of
individual species within the network. Metrics such as trophic position,
basal and top species proportions, centrality, and trophic similarity
capture how energy enters the system, which species process or
redistribute that energy, and the degree of functional redundancy among
taxa (Yodzis \& Winemiller 1999; Estrada 2007; Allesina \& Pascual
2009). At this scale, structure primarily reflects who participates in
energy transfer and whether alternative species can compensate for local
losses, a long-standing mechanism proposed to underpin persistence in
complex ecosystems (McCann 2000).

\textbf{Path-level structure}: describes how energy is routed through
sequences of interactions linking species. Metrics such as food chain
length, omnivory, motifs, loops, and prey--predator ratios capture the
coupling of energy channels and the multiplicity of pathways connecting
basal resources to higher trophic levels (Pimm \& Lawton 1977; McCann et
al.~1998; Stouffer \& Bascompte 2011). This scale reflects how energy
moves through the network and how perturbations may be transmitted
across species, with particular pathway configurations either amplifying
or diffusing disturbance effects (Neutel et al.~2002; Rooney et
al.~2006).

\textbf{Network geometry}: captures the global arrangement and
constraints of interactions, including link density, connectance,
clustering, modularity, intervality, and network distances (Dunne et
al.~2002; Stouffer et al.~2006; Delmas et al.~2019). These properties
define the overall ``shape'' of the food web and constrain which
pathways are available for energy flow. Geometric organisation has been
shown to influence the containment of perturbations, for example through
compartmentalisation that limits the spread of local disturbances
(Stouffer \& Bascompte, 2011).

\textbf{Emergent system-level behaviour}: reflects the collective
dynamical properties that arise from the interaction of nodes, paths,
and geometry. Metrics such as spectral radius, SVD complexity, and
robustness capture properties that cannot be attributed to individual
species or interactions alone, but instead describe the system's overall
capacity to absorb, reorganise, or amplify perturbations (May 1972;
Staniczenko et al.~2013; Strydom et al.~2021).

Together, these structural scales form a causal hierarchy: species roles
give rise to interaction pathways, which are embedded within a global
network geometry, from which emergent dynamical behaviour arises.

\subsubsection{Stability as a multi-component
property}\label{stability-as-a-multi-component-property}

Stability is often treated as a single outcome, yet ecological theory
has long recognised that it comprises multiple, distinct components
(Pimm 1984; Ives \& Carpenter 2007). Here, we focus on three
complementary stability mechanisms that are directly interpretable in
terms of energy flow: persistence, resistance, and return.

\textbf{Persistence}: refers to the continued ability of the system to
sustain energy flow following species loss or disturbance. Structures
that promote redundancy, multiple basal inputs, and functional overlap
among species enhance persistence by ensuring that energy can continue
to enter and circulate within the network even when individual
components are removed (McCann 2000; Dunne et al.~2002; Jonsson et
al.~2015).

\textbf{Resistance}: describes the extent to which perturbations are
attenuated or contained rather than propagating through the network.
Path-level structures such as omnivory, short chains, motifs, and
modular organisation can diffuse or localise disturbances, reducing the
likelihood that local perturbations escalate into system-wide effects
(McCann et al.~1998; Neutel et al.~2002; Stouffer \& Bascompte 2011).

\textbf{Return}: captures the capacity of the system to reorganise or
recover following disturbance, including the re-establishment of
interaction pathways and the damping of oscillations. Emergent
properties linked to global organisation and heterogeneity, such as
spectral radius and structural complexity, reflect constraints on
system-wide dynamics and influence the speed and manner with which a
system returns to equilibrium or a new stable configuration (May 1972;
Ulanowicz 2001; Staniczenko et al.~2013).

Crucially, these stability components operate at different structural
scales and are not expected to respond uniformly to the same network
properties.

\subsubsection{Implications for interpreting network
metrics}\label{implications-for-interpreting-network-metrics}

Within this framework, metrics that are often treated as competing
predictors of `stability' instead emerge as complementary descriptors of
different stability mechanisms (Thompson et al., 2012). Node-level
metrics primarily relate to persistence, path-level metrics to
resistance, and global organisational metrics to return dynamics. Some
descriptors span multiple scales, reflecting the coupling between
structural organisation and emergent behaviour (Allesina \& Tang 2012).

This perspective provides a mechanistic explanation for why studies
using different network metrics frequently report contrasting
structure--stability relationships. Rather than reflecting inconsistency
or redundancy, these differences arise because different metrics
implicitly target different components of stability (Lau et al., 2017).
By explicitly linking structural scale, energy flow, and stability
mechanism, this framework provides a principled basis for interpreting
network metrics and for selecting descriptors that align with specific
ecological questions.

\begin{longtable}[]{@{}
  >{\raggedright\arraybackslash}p{(\linewidth - 6\tabcolsep) * \real{0.2500}}
  >{\raggedright\arraybackslash}p{(\linewidth - 6\tabcolsep) * \real{0.2500}}
  >{\raggedright\arraybackslash}p{(\linewidth - 6\tabcolsep) * \real{0.2500}}
  >{\raggedright\arraybackslash}p{(\linewidth - 6\tabcolsep) * \real{0.2500}}@{}}
\caption{Words}\tabularnewline
\toprule\noalign{}
\begin{minipage}[b]{\linewidth}\raggedright
Structural scale
\end{minipage} & \begin{minipage}[b]{\linewidth}\raggedright
What it encodes
\end{minipage} & \begin{minipage}[b]{\linewidth}\raggedright
Energy-flow interpretation
\end{minipage} & \begin{minipage}[b]{\linewidth}\raggedright
Stability component
\end{minipage} \\
\midrule\noalign{}
\endfirsthead
\toprule\noalign{}
\begin{minipage}[b]{\linewidth}\raggedright
Structural scale
\end{minipage} & \begin{minipage}[b]{\linewidth}\raggedright
What it encodes
\end{minipage} & \begin{minipage}[b]{\linewidth}\raggedright
Energy-flow interpretation
\end{minipage} & \begin{minipage}[b]{\linewidth}\raggedright
Stability component
\end{minipage} \\
\midrule\noalign{}
\endhead
\bottomrule\noalign{}
\endlastfoot
Node & Species roles and redundancy & Who handles energy & Persistence
(can energy still enter and move?) \\
Path & Energy routing and coupling & How energy moves & Resistance (does
disturbance spread?) \\
Geometry & Network constraints and organisation & Where energy can go &
Containment / buffering \\
Behaviour & Emergent dynamics & How energy reorganizes & Return /
reassembly \\
\end{longtable}

\section{Materials \& Methods}\label{materials-methods}

\subsection{Data Compilation}\label{data-compilation}

We compiled quantitative network data from XX, resulting in a total of
XX ecological networks. Each network was characterized using a suite of
XX structural metrics Table~\ref{tbl-properties}, including classic
descriptors such as richness, connectance, and modularity, as well as
information-theoretic measures like spectral radius and SVD complexity.
Prior to analysis, networks with missing values were omitted, and all
metrics were standardized (mean = 0, SD = 1) to account for differences
in scale and units across descriptors.

\begin{longtable}[]{@{}
  >{\raggedright\arraybackslash}p{(\linewidth - 6\tabcolsep) * \real{0.2297}}
  >{\raggedright\arraybackslash}p{(\linewidth - 6\tabcolsep) * \real{0.2297}}
  >{\raggedright\arraybackslash}p{(\linewidth - 6\tabcolsep) * \real{0.3108}}
  >{\raggedright\arraybackslash}p{(\linewidth - 6\tabcolsep) * \real{0.2297}}@{}}
\caption{An informative caption about the different network properties.
We use a combination of metrics from both the original Vermaat et al.
(2009) paper as well as including those that have been identified by
Thompson et al. (2012) and have been linked to emerging ecosystem
properties such as stability}\label{tbl-properties}\tabularnewline
\toprule\noalign{}
\begin{minipage}[b]{\linewidth}\raggedright
Label
\end{minipage} & \begin{minipage}[b]{\linewidth}\raggedright
Definition
\end{minipage} & \begin{minipage}[b]{\linewidth}\raggedright
Structural interpretation
\end{minipage} & \begin{minipage}[b]{\linewidth}\raggedright
Reference
\end{minipage} \\
\midrule\noalign{}
\endfirsthead
\toprule\noalign{}
\begin{minipage}[b]{\linewidth}\raggedright
Label
\end{minipage} & \begin{minipage}[b]{\linewidth}\raggedright
Definition
\end{minipage} & \begin{minipage}[b]{\linewidth}\raggedright
Structural interpretation
\end{minipage} & \begin{minipage}[b]{\linewidth}\raggedright
Reference
\end{minipage} \\
\midrule\noalign{}
\endhead
\bottomrule\noalign{}
\endlastfoot
Basal & Proportion of taxa with zero vulnerability (no consumers). &
Quantifies the proportion of species representing basal energy inputs to
the network. & \\
Top & Proportion of taxa with zero generality (no resources). &
Describes the relative prevalence of terminal consumers in the network.
& \\
Intermediate & Proportion of taxa with both consumers and resources. &
Captures the proportion of species participating in both upward and
downward energy transfer. & \\
Richness (S) & Number of taxa (nodes) in the network. & Describes
network size. & \\
Links (L) & Total number of trophic interactions (edges). & Describes
interaction density independent of network size. & \\
Connectance & \(L/S^2\), where \(S\) is the number of species and \(L\)
the number of links & Measures the proportion of realised interactions
relative to all possible interactions. & Dunne et al.~2002 \\
L/S & Mean number of links per species. & Captures average interaction
density per taxon. & \\
Cannibal & Proportion of taxa with self-loops. & Quantifies the
prevalence of cannibalistic interactions. & \\
Herbivore & Proportion of taxa feeding exclusively on basal species. &
Describes the representation of primary consumers. & \\
Intermediate & Percentage of intermediate taxa (with both consumers and
resources) & & \\
Trophic level (TL) & Prey-weighted trophic level averaged across taxa. &
Captures the vertical organisation of energy transfer. & Williams \&
Martinez (2004) \\
MaxSim & Mean maximum trophic similarity of each taxon to all others. &
Quantifies functional similarity based on shared predators and prey. &
Yodzis \& Winemiller (1999) \\
Centrality & Node centrality averaged across taxa
(definition-dependent). & Captures the distribution of influence or
connectivity among species. & Estrada \& Bodin (2008) \\
ChLen & Mean length of all food chains from basal to top taxa. &
Describes the average number of steps in energy-transfer pathways. & \\
ChSD & Standard deviation of food chain length. & Captures variability
in pathway lengths. & \\
ChNum & Log-transformed number of distinct food chains. & Quantifies the
multiplicity of alternative energy pathways. & \\
Path & Mean shortest path length between all species pairs. & Describes
the average distance between taxa within the network. & \\
Diameter & Maximum shortest path length between any two taxa. & Captures
the largest network distance between species. & \\
Omnivory & Proportion of taxa feeding on resources at multiple trophic
levels. & Describes vertical coupling of energy channels. & McCann
(2000) \\
Loop & Proportion of taxa involved in trophic loops. & Quantifies the
prevalence of cyclic interaction pathways. & \\
Prey:Predator & Ratio of prey taxa (basal + intermediate) to predator
taxa (intermediate + top). & Describes the overall shape of the trophic
structure. & \\
Diameter & Diameter can also be measured as the average of the distances
between each pair of nodes in the network & & \\
Clust & Mean clustering coefficient. & Measures the tendency for taxa
sharing interaction partners to also interact with each other. & Watts
\& Strogatz (1998) \\
GenSD & Normalised standard deviation of generality. & Captures
heterogeneity in the number of resources per taxon. & Williams \&
Martinez (2004) \\
VulSD & Normalised standard deviation of vulnerability. & Captures
heterogeneity in the number of consumers per taxon. & Williams \&
Martinez (2004) \\
LinkSD & Normalised standard deviation of total links per taxon. &
Quantifies variation in species connectivity. & \\
Intervality & Degree to which taxa can be ordered along a single niche
dimension. & Measures the extent of niche ordering in trophic
interactions. & Stouffer et al. (2006) \\
ρ (Spectral radius) & Largest real part of the eigenvalues of the
undirected adjacency matrix. & Captures a global property of network
organisation related to interaction strength aggregation. & \\
Complexity (SVD) & Shannon entropy of the singular value decomposition
of the adjacency matrix. & Quantifies heterogeneity in interaction
structure. & Strydom et al. (2021) \\
Robustness & Proportion of secondary extinctions following primary
species removal. & Operational measure of tolerance to node loss. &
Jonsson et al. (2015) \\
S1 (Linear chain) & Frequency of three-node linear chains (A → B → C)
with no additional links. & Captures the prevalence of simple,
unbranched energy-transfer pathways. & Stouffer et al. (2007) Milo et
al. (2002) \\
S2 (Omnivory) & Frequency of three-node motifs forming a feed-forward
loop (A → B → C, A → C). & Describes vertical coupling of trophic levels
within small subnetworks. & Stouffer et al. (2007) Milo et al. (2002) \\
S4 (Apparent competition) & Frequency of motifs where one consumer feeds
on two resources (A → B ← C). & Captures the prevalence of
shared-predator structures among resources. & Stouffer et al. (2007)
Milo et al. (2002) \\
S5 (Direct competition) & Frequency of motifs where two consumers share
a single resource (A ← B → C). & Describes the occurrence of
shared-resource structures among consumers. & Stouffer et al. (2007)
Milo et al. (2002) \\
\end{longtable}

\subsection{Statistical Characterisation of Network
Space}\label{statistical-characterisation-of-network-space}

To identify the dominant axes of structural variation within our
dataset, we expanded upon the dimensionality reduction approach
established by Vermaat et al.~(2009). We quantified 31 structural
metrics for each food web, categorising them into four hierarchical
scales of energy flow: Node-level (e.g., trophic roles), Path-level
(e.g., energy sequences), Network Geometry (e.g., global shape), and
Emergent Behaviour (e.g., system-wide complexity).

\subsubsection{Dimensionality Reduction and Metric
Selection}\label{dimensionality-reduction-and-metric-selection}

To identify a non-redundant subset of structural descriptors, we
performed a Principal Component Analysis (PCA) on the scaled structural
metrics. We excluded stability-related outcomes (\emph{e.g.,}
robustness, ρ, and SVD complexity) from the PCA to ensure the resulting
dimensions represented pure physical architecture.

We determined the number of significant structural dimensions using
Kaiser's Criterion (eigenvalues \textgreater1) and an analysis of the
scree plot `elbow'. To select a representative anchor for each
significant dimension, we programmatically identified the metric with
the highest contribution percentage (cos\^{}2) to each Principal
Component (PC). In cases where a single metric dominated multiple
dimensions, the next highest unique contributor was selected to ensure a
parsimonious yet comprehensive subset of predictors.

\subsection{Linking Structure to Stability
Components}\label{linking-structure-to-stability-components}

To test the hypothesis that different hierarchical structural scales
govern distinct aspects of ecological stability, we employed Random
Forest (RF) models. We defined three response variables representing
components of Stability:

\begin{itemize}
\tightlist
\item
  Persistence: Quantified as structural Robustness (secondary extinction
  resistance).
\item
  Dampening: Quantified as the Spectral Radius (ρ), representing local
  stability.
\item
  Organization: Quantified as SVD Complexity, representing the
  informational diversity of energy channels.
\end{itemize}

\subsubsection{Model Specification and
Importance}\label{model-specification-and-importance}

Each stability component was modelled as a function of the six
`Structural Pillars' identified in the PCA phase (S1, Richness,
PredPreyRatio, Distance, Herbivory, and MaxSim). We used 1,000 trees per
forest to ensure stable importance estimates. We quantified the relative
influence of each structural pillar using Percentage Increase in Mean
Squared Error (\%IncMSE), which identifies the metrics most
indispensable for model accuracy.

\subsubsection{Directional Effects}\label{directional-effects}

To interpret the ecological nature of these relationships, we generated
Partial Dependence Plots (PDPs). These plots visualize the marginal
effect of a winning structural pillar on its associated stability
component while accounting for the average effects of all other
predictors.

\subsection{Validation of Structural
Modularity}\label{validation-of-structural-modularity}

To assess the redundancy and independence of the 31 structural metrics,
we performed a Hierarchical Variable Clustering analysis using the
hclust algorithm in R. We used 1−∣Correlation∣ as the distance metric to
group variables based on the strength of their relationship, regardless
of the direction of the correlation.

To determine the optimal number of structural modules, we calculated
Average Silhouette Widths for k=2 to 10. Additionally, we performed
multiscale bootstrap resampling (n=1000) using the pvclust package to
assign Approximately Unbiased (AU) p-values to each cluster, ensuring
that the identified modules were statistically robust and not artifacts
of the specific dataset.

\section{Results}\label{results}

\subsection{The Six Dimensions of Food Web
Architecture}\label{the-six-dimensions-of-food-web-architecture}

The PCA revealed a high-dimensional structural space where the first six
principal components accounted for 84.9\% of the total structural
variation (Table 1). This indicates that while food web structure is
complex, it can be effectively distilled into six primary axes of
architectural variation.

\begin{itemize}
\tightlist
\item
  Dimension 1 (31.6\% variance) was anchored by S1 (the number of
  pathways from basal species), representing the total volume of energy
  entry points.
\item
  Dimension 2 (21.6\% variance) was dominated by Richness (S),
  representing the overall scale of the network.
\item
  Dimension 3 (15.5\% variance) was defined by the Predator-Prey Ratio,
  capturing the balance of trophic roles (Node-level).
\item
  Dimension 4 (6.8\% variance) was driven by Mean Distance, reflecting
  the efficiency or tightness of the wiring (Path-level).
\item
  Dimension 5 (5.2\% variance) was anchored by Herbivory, identifying
  the functional dominance of specific energy channels.
\item
  Dimension 6 (4.4\% variance) was represented by MaxSim (Maximum
  Similarity), a measure of trophic redundancy and functional overlap.
\end{itemize}

\subsection{Alignment with the Energy-Flow
Hierarchy}\label{alignment-with-the-energy-flow-hierarchy}

The distribution of these representative metrics across the PCA
dimensions suggests that no single hierarchical scale dominates food-web
variation. Instead, the network space is composed of a mixture of
Path-level (S1, Distance), Geometry (Richness), and Node-level
(PredPreyRatio, Herbivory, MaxSim) properties. Notably, S1 (a Path-level
metric) explained more variance than Richness, suggesting that the
arrangement of energy flow pathways is a more fundamental structural
signature than simple species counts.

\subsubsection{Differential Drivers of Ecological
Stability}\label{differential-drivers-of-ecological-stability}

The Random Forest models revealed a clear partitioning of structural
influence, with different `pillars' emerging as the dominant predictors
for each stability component (Table 2).

Persistence (Robustness) was primarily governed by PredPreyRatio
(\%IncMSE = 15.98) and Herbivory (\%IncMSE = 10.23). Surprisingly,
network Richness had no predictive power for persistence (\%IncMSE =
-2.49), suggesting that the balance of trophic roles is more critical
for preventing extinction cascades than the total number of species.

Dampening (ρ) was most sensitive to Herbivory (\%IncMSE = 17.28). This
indicates that the strength and proportion of energy flow originating
from the basal-consumer interface is the primary regulator of a
network's local stability.

Organization (SVD Complexity) was uniquely driven by MaxSim (\%IncMSE =
13.12) and S1 (\%IncMSE = 12.56). This supports the view that
informational complexity emerges from a combination of unique species
roles (low similarity) and the total volume of energy entry pathways.

\subsubsection{Directionality and Trophic
Control}\label{directionality-and-trophic-control}

Partial dependence analysis provided a mechanistic look at how these
pillars operate:

Trophic Balance: Persistence showed a {[}direction{]} response to the
PredPreyRatio, indicating an optimal balance between consumers and
resources is required to maintain structural integrity.

Basal Channels: The system's ability to dampen perturbations (ρ)
improved as Herbivory {[}increased/decreased{]}, suggesting that
basal-heavy webs act as a buffer for energy surges.

Specialization: SVD Complexity increased as MaxSim decreased, confirming
that organised networks are those where species occupy distinct,
non-overlapping trophic niches rather than redundant ones.

Our multi-stage analysis reveals that food web stability is governed by
a hierarchical architecture where structural modules serve as
specialized functional levers. While Principal Component Analysis and
Silhouette validation (k=10) confirm the high dimensionality of network
space, the hierarchical clustering identifies four robust, independent
modules (AU ≥ 0.94) that dictate system behaviour. We find a clear
decoupling of network scale from network persistence; specifically, the
Geometric Scale module (containing Richness and Connectance) accounts
for significant structural variance but fails to predict any component
of stability. Instead, the Energy Flow Engine (Module 3) and Trophic
Redundancy (Module 4) emerge as the primary controllers. Specifically,
node-level balance within the flow engine (PredPreyRatio and Herbivory)
determines persistence and dampening, while the isolation of species
roles (MaxSim) dictates organizational complexity. This suggests that
the stability of an ecosystem is not a product of its size, but of the
specific configuration of its energy channels and the redundancy of its
functional roles.

\begin{itemize}
\tightlist
\item
  Redundancy / persistence
\item
  Pathway coupling / resistance
\item
  Organizational constraint / return
\item
  Reduced metric sets retain these axes
\end{itemize}

\section{Discussion}\label{discussion}

\subsection{Why consensus on ``the best metric'' has remained
elusive}\label{why-consensus-on-the-best-metric-has-remained-elusive}

For more than four decades, the structure--stability debate has
oscillated between competing structural descriptors, from connectance
and complexity to modularity, omnivory, and trophic redundancy. Early
theoretical work, most notably by May, suggested that increasing
complexity should destabilise large systems, placing global descriptors
such as connectance and interaction density at the centre of the debate.
Subsequent empirical and theoretical studies, however, demonstrated
stabilising roles for omnivory, weak links, and compartmentalisation
(e.g.~McCann; Dunne), shifting attention toward pathway-level and
geometric properties. Our results suggest that this apparent
inconsistency does not arise because some metrics are ``wrong,'' but
because they capture different dimensions of a fundamentally
multi-layered architecture. Food-web structure is not a single axis
ranging from simple to complex; it is a high-dimensional space in which
node composition, pathway routing, geometric constraint, and emergent
organisation vary semi-independently. Metrics that have historically
been treated as competitors---such as richness, connectance, omnivory,
and redundancy---are in fact anchors of distinct structural dimensions.
This explains why no single metric consistently predicts ``stability''
across studies: stability itself is not singular. When studies measure
persistence (e.g.~secondary extinctions), node-level balance and
redundancy dominate. When they measure local stability or damping,
path-level coupling and basal channel dominance matter more. When they
measure recovery or informational organisation, global heterogeneity and
role differentiation become decisive. The search for a universal
structural predictor of stability has therefore been misguided---not
because structure is unimportant, but because both structure and
stability are multidimensional.

\subsection{Reconciling contrasting structure--stability
relationships}\label{reconciling-contrasting-structurestability-relationships}

A persistent challenge in the literature has been that similar metrics
sometimes show opposite associations with stability. For example,
omnivory has been reported as stabilising in some contexts and
destabilising in others. Within our framework, this variability is
expected. Path-level coupling (e.g.~omnivory motifs) primarily relates
to resistance---the containment or diffusion of perturbations---rather
than persistence per se. A structure that dampens oscillations may not
prevent extinction cascades, and vice versa. Similarly, our finding that
richness contributes substantially to structural variance but fails to
predict any stability component underscores an important decoupling:
network size is not synonymous with robustness. This result aligns with
empirical syntheses showing that species richness alone poorly predicts
secondary extinction resistance unless accompanied by trophic balance or
redundancy. In our analyses, the predator--prey ratio and
herbivory---metrics reflecting trophic shape and basal channel
dominance---were far stronger determinants of persistence and damping
than richness itself. This helps reconcile why biodiversity--stability
studies sometimes find positive, neutral, or context-dependent
relationships. Richness operates primarily as a geometric scaling
variable. Its effect on stability depends on how additional species
alter the configuration of energy channels---whether they add
redundancy, elongate chains, skew trophic balance, or increase coupling.
Without specifying which structural dimension richness modifies, its
stability effect is indeterminate.

\subsection{Implications for biodiversity--stability
theory}\label{implications-for-biodiversitystability-theory}

Our results contribute to a long-standing tension between the
``complexity begets instability'' argument and more recent views that
diversity enhances resilience. The resolution lies in recognising that
different dimensions of complexity influence different stability
components. Redundancy and trophic similarity (low MaxSim heterogeneity)
enhance persistence by providing functional compensation following
species loss. Pathway multiplicity and basal channel dominance regulate
resistance by shaping how perturbations propagate. Global organisational
heterogeneity (captured by SVD complexity and spectral properties)
influences return dynamics by constraining system-wide oscillations and
reassembly patterns. In this sense, complexity is neither inherently
stabilising nor destabilising. Instead, distinct forms of complexity map
onto distinct dynamical outcomes. Connectance, motif structure,
redundancy, and informational heterogeneity are not interchangeable
proxies for ``complexity,'' but structurally and mechanistically
distinct attributes. By explicitly embedding food-web metrics within an
energy-flow hierarchy, our framework reframes biodiversity--stability
theory. Stability emerges not from species number alone, but from the
configuration of energy channels and the distribution of functional
roles across those channels. A bottom-heavy web with balanced
predator--prey ratios and differentiated trophic niches may achieve both
persistence and damping, even at moderate richness. Conversely, a
species-rich but top-heavy or weakly differentiated web may remain
fragile.

\subsection{Structural modularity and the minimum sufficient
set}\label{structural-modularity-and-the-minimum-sufficient-set}

A key practical outcome of our analysis is the identification of six
structural pillars that capture nearly 85\% of the variation in network
architecture. Despite starting with 31 commonly used metrics, much of
their variance collapses into a smaller number of independent axes.
Hierarchical clustering further revealed robust modules corresponding to
trophic balance, pathway structure, geometric scale, and redundancy.
This dimensional reduction has two important implications. First, it
demonstrates that many commonly reported metrics are statistically
redundant. Reporting both connectance and L/S, or multiple correlated
distance measures, adds little new information. Second, it provides a
principled route toward a minimum sufficient descriptor set tailored to
ecological questions. Rather than selecting metrics by convention,
researchers can select representatives of the relevant structural scale.

\subsection{Guidance for metric choice based on ecological
question}\label{guidance-for-metric-choice-based-on-ecological-question}

If the question concerns extinction cascades or tolerance to species
loss (persistence): prioritise node-level balance and redundancy metrics
(e.g.~predator--prey ratio, herbivory, similarity). If the question
concerns perturbation propagation or damping (resistance): focus on
path-level and basal-channel metrics (e.g.~pathway multiplicity,
omnivory, mean distance). If the question concerns recovery dynamics or
organisational reassembly (return): emphasise global heterogeneity and
spectral measures (e.g.~SVD complexity, spectral radius). If the
question concerns comparative architecture across systems: use geometric
anchors (e.g.~richness) but interpret them as scale descriptors rather
than stability predictors. This scale-explicit approach shifts the
emphasis from asking ``Which metric best predicts stability?'' to
``Which structural dimension corresponds to the stability mechanism of
interest?''

\subsection{Limitations and future
directions}\label{limitations-and-future-directions}

While our framework clarifies structural dimensionality, it remains
grounded in static network topology. Real ecosystems exhibit interaction
strengths, temporal variability, adaptive rewiring, and environmental
forcing. Incorporating weighted networks and dynamic simulations would
strengthen the mechanistic link between structural axes and realised
dynamics. Additionally, the separation of persistence, resistance, and
return is analytically useful but biologically intertwined. For example,
structures that enhance resistance may indirectly support persistence by
preventing large cascades. Future work could explore how these
components interact across environmental gradients and disturbance
regimes. Finally, extending this framework beyond trophic networks to
mutualistic, host--parasite, or multilayer networks would test its
generality. The energy-flow hierarchy proposed here may represent a
broader organising principle of ecological interaction systems.

\section{Conclusion}\label{conclusion}

Food-web stability is not governed by a single structural property, but
by multiple, hierarchically organised dimensions of network
architecture. Node composition, pathway routing, geometric constraint,
and emergent organisation each represent distinct levers through which
ecosystems maintain persistence, resist perturbation, and reorganise
following disturbance.

By explicitly mapping structural scale to stability mechanism, we
reconcile decades of seemingly conflicting findings in the
structure--stability literature. The key insight is not that one metric
is superior, but that different metrics answer different ecological
questions. Recognising and embracing this multidimensionality provides a
coherent foundation for future biodiversity--stability research and a
practical framework for selecting network descriptors based on mechanism
rather than tradition.

\begin{longtable}[]{@{}
  >{\raggedright\arraybackslash}p{(\linewidth - 6\tabcolsep) * \real{0.2361}}
  >{\raggedright\arraybackslash}p{(\linewidth - 6\tabcolsep) * \real{0.2361}}
  >{\raggedright\arraybackslash}p{(\linewidth - 6\tabcolsep) * \real{0.2917}}
  >{\raggedright\arraybackslash}p{(\linewidth - 6\tabcolsep) * \real{0.2361}}@{}}
\caption{Network properties used for
analysis.}\label{tbl-metrics}\tabularnewline
\toprule\noalign{}
\begin{minipage}[b]{\linewidth}\raggedright
Dimension
\end{minipage} & \begin{minipage}[b]{\linewidth}\raggedright
Key Metrics
\end{minipage} & \begin{minipage}[b]{\linewidth}\raggedright
Expected Effect on Stability
\end{minipage} & \begin{minipage}[b]{\linewidth}\raggedright
Supporting Literature
\end{minipage} \\
\midrule\noalign{}
\endfirsthead
\toprule\noalign{}
\begin{minipage}[b]{\linewidth}\raggedright
Dimension
\end{minipage} & \begin{minipage}[b]{\linewidth}\raggedright
Key Metrics
\end{minipage} & \begin{minipage}[b]{\linewidth}\raggedright
Expected Effect on Stability
\end{minipage} & \begin{minipage}[b]{\linewidth}\raggedright
Supporting Literature
\end{minipage} \\
\midrule\noalign{}
\endhead
\bottomrule\noalign{}
\endlastfoot
Complexity \& Redundancy & Connectance, MaxSim, Links &
\textbf{Positive:} High redundancy allows for `functional compensation'
if one species is lost. & Dunne et al. (2002); McCann (2000) \\
Compartmentalization & Clust, Modularity,~ρ & \textbf{Positive:} Limits
the spread of perturbations; local collapses don't become global. &
Stouffer \& Bascompte (2011) \\
Feedback \& Coupling & Omnivory (S2), Loop, ChLen & \textbf{Variable:}
Omnivory can stabilize by diffusing energy, but long chains can amplify
oscillations. & McCann (2000); Neutel et al. (2002) \\
Hierarchy \& Shape & Prey:Predator, Basal, Top & \textbf{Critical:}
`Bottom-heavy' systems are generally more stable; inverted pyramids are
fragile. & \\
Information Heterogeneity & SVD Complexity, LinkSD & \textbf{Positive:}
Diverse interaction strengths prevent `resonant' instabilities. &
Ulanowicz (2001) \\
\end{longtable}

To better link network structure to ecosystem function, we grouped
commonly used food web metrics into four functional categories
reflecting their ecological interpretation. Node-level metrics (e.g.,
basal, top, intermediate, herbivory, cannibal, trophic level,
centrality, MaxSim) capture the roles and properties of individual
species. Path-level metrics (\emph{e.g.,} food chain lengths, motifs,
omnivory, loops, prey-to-predator ratios) describe how energy and
interactions flow through the network. Geometry metrics (e.g.,
connectance, link density, clustering, variation in links, diameter,
intervality, spectral radius, SVD complexity) quantify the overall
network arrangement and topological organization. Finally,
behavioural/system-level metrics (e.g., robustness, spectral radius, SVD
complexity) capture emergent properties such as resilience, redundancy,
and dynamic stability. This framework provides a transparent,
ecologically interpretable mapping from species-level roles and
interaction paths to system-level behaviour, facilitates identification
of a Minimum Sufficient Set of descriptors, and links network structure
to long-standing questions in the stability--complexity debate.

\begin{longtable}[]{@{}
  >{\raggedright\arraybackslash}p{(\linewidth - 4\tabcolsep) * \real{0.3333}}
  >{\raggedright\arraybackslash}p{(\linewidth - 4\tabcolsep) * \real{0.3333}}
  >{\raggedright\arraybackslash}p{(\linewidth - 4\tabcolsep) * \real{0.3333}}@{}}
\caption{Classification of food web metrics into functional categories.
Node-level metrics capture species roles; path-level metrics capture
energy or interaction flows; geometry metrics describe network
structure; and behaviour/system-level metrics capture emergent stability
and resilience.}\label{tbl-classification}\tabularnewline
\toprule\noalign{}
\begin{minipage}[b]{\linewidth}\raggedright
Category
\end{minipage} & \begin{minipage}[b]{\linewidth}\raggedright
Metrics
\end{minipage} & \begin{minipage}[b]{\linewidth}\raggedright
Rationale
\end{minipage} \\
\midrule\noalign{}
\endfirsthead
\toprule\noalign{}
\begin{minipage}[b]{\linewidth}\raggedright
Category
\end{minipage} & \begin{minipage}[b]{\linewidth}\raggedright
Metrics
\end{minipage} & \begin{minipage}[b]{\linewidth}\raggedright
Rationale
\end{minipage} \\
\midrule\noalign{}
\endhead
\bottomrule\noalign{}
\endlastfoot
Node-level & basal, top, intermediate, herbivory, cannibal, TL,
centrality, MaxSim & Metrics describing individual species and their
roles in the network, including trophic position (TL), feeding type
(basal, herbivory, cannibal), influence (centrality), and functional
redundancy (MaxSim). \\
Path level & ChLen, ChSD, ChNum, path, S1, S2, S4, S5, omnivory, loops,
predpreyRatio, distance & Metrics describing energy flow or interaction
paths through the network, including food chain lengths, motifs,
omnivory, loops, and prey-predator ratios. Capture how energy and
interactions move across nodes. \\
Geometry/topology & connectance, l\_S, links, richness, Clust, GenSD,
VulSD, LinkSD, diameter, intervals & Metrics describing~\textbf{overall
network structure}, including density (connectance, links, L/S), size
(richness), local clustering, link asymmetry, network distances
(diameter), and niche structure (intervals). Capture the `shape' of the
network. \\
Behaviour/system-level & ρ, complexity, robustness & Metrics describing
emergent properties of the network, including system-wide stability,
resilience, and structural complexity. Capture how the network responds
to perturbations or organizes itself at a global level. \\
\end{longtable}

{

\begin{longtable}[]{@{}llrr@{}}

\caption{\label{tbl-corr}Here is a table showing the correlation of the
different network properties with the first three dimensions of the PCA}

\tabularnewline

\toprule\noalign{}
Property & dimension & quanti.correlation & quanti.p.value \\
\midrule\noalign{}
\endhead
\bottomrule\noalign{}
\endlastfoot
richness & Dim.1 & 0.38 & NA \\
richness & Dim.2 & 0.84 & NA \\
richness & Dim.3 & -0.27 & NA \\
richness & Dim.4 & 0.04 & NA \\
richness & Dim.5 & -0.05 & NA \\
links & Dim.1 & 0.68 & NA \\
links & Dim.2 & 0.68 & NA \\
links & Dim.3 & 0.01 & NA \\
links & Dim.4 & 0.03 & NA \\
links & Dim.5 & 0.12 & NA \\
connectance & Dim.1 & 0.40 & NA \\
connectance & Dim.2 & -0.56 & NA \\
connectance & Dim.3 & 0.60 & NA \\
connectance & Dim.4 & -0.04 & NA \\
connectance & Dim.5 & 0.10 & NA \\
diameter & Dim.1 & 0.80 & NA \\
diameter & Dim.2 & 0.27 & NA \\
diameter & Dim.3 & -0.24 & NA \\
diameter & Dim.4 & 0.02 & NA \\
diameter & Dim.5 & 0.11 & NA \\
distance & Dim.1 & -0.01 & NA \\
distance & Dim.2 & 0.33 & NA \\
distance & Dim.3 & 0.08 & NA \\
distance & Dim.4 & 0.59 & NA \\
distance & Dim.5 & 0.40 & NA \\
basal & Dim.1 & -0.52 & NA \\
basal & Dim.2 & 0.44 & NA \\
basal & Dim.3 & 0.65 & NA \\
basal & Dim.4 & 0.22 & NA \\
basal & Dim.5 & -0.19 & NA \\
top & Dim.1 & -0.54 & NA \\
top & Dim.2 & 0.21 & NA \\
top & Dim.3 & -0.38 & NA \\
top & Dim.4 & -0.55 & NA \\
top & Dim.5 & 0.09 & NA \\
intermediate & Dim.1 & 0.72 & NA \\
intermediate & Dim.2 & -0.48 & NA \\
intermediate & Dim.3 & -0.35 & NA \\
intermediate & Dim.4 & 0.10 & NA \\
intermediate & Dim.5 & 0.11 & NA \\
predpreyRatio & Dim.1 & -0.31 & NA \\
predpreyRatio & Dim.2 & 0.39 & NA \\
predpreyRatio & Dim.3 & 0.71 & NA \\
predpreyRatio & Dim.4 & 0.31 & NA \\
predpreyRatio & Dim.5 & -0.27 & NA \\
herbivory & Dim.1 & -0.51 & NA \\
herbivory & Dim.2 & 0.27 & NA \\
herbivory & Dim.3 & 0.02 & NA \\
herbivory & Dim.4 & -0.17 & NA \\
herbivory & Dim.5 & 0.55 & NA \\
omnivory & Dim.1 & 0.77 & NA \\
omnivory & Dim.2 & -0.31 & NA \\
omnivory & Dim.3 & -0.14 & NA \\
omnivory & Dim.4 & 0.05 & NA \\
omnivory & Dim.5 & -0.15 & NA \\
cannibal & Dim.1 & 0.69 & NA \\
cannibal & Dim.2 & 0.07 & NA \\
cannibal & Dim.3 & 0.40 & NA \\
cannibal & Dim.4 & -0.39 & NA \\
cannibal & Dim.5 & 0.00 & NA \\
l\_S & Dim.1 & 0.84 & NA \\
l\_S & Dim.2 & 0.45 & NA \\
l\_S & Dim.3 & 0.26 & NA \\
l\_S & Dim.4 & -0.09 & NA \\
l\_S & Dim.5 & 0.03 & NA \\
GenSD & Dim.1 & -0.39 & NA \\
GenSD & Dim.2 & 0.67 & NA \\
GenSD & Dim.3 & 0.33 & NA \\
GenSD & Dim.4 & 0.21 & NA \\
GenSD & Dim.5 & -0.32 & NA \\
VulSD & Dim.1 & -0.34 & NA \\
VulSD & Dim.2 & 0.56 & NA \\
VulSD & Dim.3 & -0.42 & NA \\
VulSD & Dim.4 & -0.46 & NA \\
VulSD & Dim.5 & 0.17 & NA \\
TL & Dim.1 & 0.57 & NA \\
TL & Dim.2 & -0.40 & NA \\
TL & Dim.3 & -0.66 & NA \\
TL & Dim.4 & -0.05 & NA \\
TL & Dim.5 & -0.08 & NA \\
ChLen & Dim.1 & 0.54 & NA \\
ChLen & Dim.2 & -0.54 & NA \\
ChLen & Dim.3 & -0.49 & NA \\
ChLen & Dim.4 & 0.12 & NA \\
ChLen & Dim.5 & -0.16 & NA \\
ChSD & Dim.1 & 0.37 & NA \\
ChSD & Dim.2 & 0.11 & NA \\
ChSD & Dim.3 & -0.45 & NA \\
ChSD & Dim.4 & 0.40 & NA \\
ChSD & Dim.5 & -0.10 & NA \\
ChNum & Dim.1 & -0.11 & NA \\
ChNum & Dim.2 & 0.76 & NA \\
ChNum & Dim.3 & -0.47 & NA \\
ChNum & Dim.4 & -0.24 & NA \\
ChNum & Dim.5 & -0.15 & NA \\
path & Dim.1 & 0.31 & NA \\
path & Dim.2 & 0.34 & NA \\
path & Dim.3 & -0.36 & NA \\
path & Dim.4 & 0.54 & NA \\
path & Dim.5 & 0.35 & NA \\
LinkSD & Dim.1 & -0.19 & NA \\
LinkSD & Dim.2 & 0.72 & NA \\
LinkSD & Dim.3 & -0.39 & NA \\
LinkSD & Dim.4 & -0.05 & NA \\
LinkSD & Dim.5 & -0.29 & NA \\
S1 & Dim.1 & 0.90 & NA \\
S1 & Dim.2 & -0.03 & NA \\
S1 & Dim.3 & 0.14 & NA \\
S1 & Dim.4 & -0.07 & NA \\
S1 & Dim.5 & 0.08 & NA \\
S2 & Dim.1 & 0.79 & NA \\
S2 & Dim.2 & -0.06 & NA \\
S2 & Dim.3 & 0.47 & NA \\
S2 & Dim.4 & -0.26 & NA \\
S2 & Dim.5 & -0.06 & NA \\
S4 & Dim.1 & 0.62 & NA \\
S4 & Dim.2 & 0.48 & NA \\
S4 & Dim.3 & 0.28 & NA \\
S4 & Dim.4 & -0.26 & NA \\
S4 & Dim.5 & 0.18 & NA \\
S5 & Dim.1 & 0.66 & NA \\
S5 & Dim.2 & 0.42 & NA \\
S5 & Dim.3 & 0.52 & NA \\
S5 & Dim.4 & -0.08 & NA \\
S5 & Dim.5 & 0.00 & NA \\
centrality & Dim.1 & -0.30 & NA \\
centrality & Dim.2 & -0.61 & NA \\
centrality & Dim.3 & 0.26 & NA \\
centrality & Dim.4 & 0.07 & NA \\
centrality & Dim.5 & 0.46 & NA \\
loops & Dim.1 & 0.81 & NA \\
loops & Dim.2 & 0.28 & NA \\
loops & Dim.3 & 0.17 & NA \\
loops & Dim.4 & -0.09 & NA \\
loops & Dim.5 & 0.12 & NA \\
intervals & Dim.1 & 0.51 & NA \\
intervals & Dim.2 & 0.65 & NA \\
intervals & Dim.3 & -0.11 & NA \\
intervals & Dim.4 & 0.24 & NA \\
intervals & Dim.5 & 0.18 & NA \\
MaxSim & Dim.1 & -0.12 & NA \\
MaxSim & Dim.2 & -0.07 & NA \\
MaxSim & Dim.3 & 0.57 & NA \\
MaxSim & Dim.4 & -0.14 & NA \\
MaxSim & Dim.5 & 0.21 & NA \\
Clust & Dim.1 & 0.64 & NA \\
Clust & Dim.2 & -0.35 & NA \\
Clust & Dim.3 & 0.16 & NA \\
Clust & Dim.4 & 0.01 & NA \\
Clust & Dim.5 & -0.40 & NA \\
S1 & Dim.1 & 0.90 & 0.0000000 \\
l\_S & Dim.1 & 0.84 & 0.0000000 \\
loops & Dim.1 & 0.81 & 0.0000000 \\
diameter & Dim.1 & 0.80 & 0.0000000 \\
S2 & Dim.1 & 0.79 & 0.0000000 \\
omnivory & Dim.1 & 0.77 & 0.0000000 \\
intermediate & Dim.1 & 0.72 & 0.0000004 \\
cannibal & Dim.1 & 0.69 & 0.0000015 \\
links & Dim.1 & 0.68 & 0.0000030 \\
S5 & Dim.1 & 0.66 & 0.0000077 \\
Clust & Dim.1 & 0.64 & 0.0000170 \\
S4 & Dim.1 & 0.62 & 0.0000282 \\
TL & Dim.1 & 0.57 & 0.0001879 \\
ChLen & Dim.1 & 0.54 & 0.0005272 \\
intervals & Dim.1 & 0.51 & 0.0009658 \\
connectance & Dim.1 & 0.40 & 0.0123944 \\
richness & Dim.1 & 0.38 & 0.0182160 \\
ChSD & Dim.1 & 0.37 & 0.0205346 \\
VulSD & Dim.1 & -0.34 & 0.0348758 \\
GenSD & Dim.1 & -0.39 & 0.0150597 \\
herbivory & Dim.1 & -0.51 & 0.0010542 \\
basal & Dim.1 & -0.52 & 0.0009130 \\
top & Dim.1 & -0.54 & 0.0004175 \\
richness & Dim.2 & 0.84 & 0.0000000 \\
ChNum & Dim.2 & 0.76 & 0.0000000 \\
LinkSD & Dim.2 & 0.72 & 0.0000003 \\
links & Dim.2 & 0.68 & 0.0000030 \\
GenSD & Dim.2 & 0.67 & 0.0000051 \\
intervals & Dim.2 & 0.65 & 0.0000111 \\
VulSD & Dim.2 & 0.56 & 0.0002484 \\
S4 & Dim.2 & 0.48 & 0.0020815 \\
l\_S & Dim.2 & 0.45 & 0.0049841 \\
basal & Dim.2 & 0.44 & 0.0062051 \\
S5 & Dim.2 & 0.42 & 0.0091132 \\
predpreyRatio & Dim.2 & 0.39 & 0.0142971 \\
path & Dim.2 & 0.34 & 0.0381102 \\
distance & Dim.2 & 0.33 & 0.0426262 \\
Clust & Dim.2 & -0.35 & 0.0303883 \\
TL & Dim.2 & -0.40 & 0.0139488 \\
intermediate & Dim.2 & -0.48 & 0.0023996 \\
ChLen & Dim.2 & -0.54 & 0.0005074 \\
connectance & Dim.2 & -0.56 & 0.0002624 \\
centrality & Dim.2 & -0.61 & 0.0000504 \\
predpreyRatio & Dim.3 & 0.71 & 0.0000006 \\
basal & Dim.3 & 0.65 & 0.0000122 \\
connectance & Dim.3 & 0.60 & 0.0000804 \\
MaxSim & Dim.3 & 0.57 & 0.0002078 \\
S5 & Dim.3 & 0.52 & 0.0008330 \\
S2 & Dim.3 & 0.47 & 0.0029951 \\
cannibal & Dim.3 & 0.40 & 0.0122774 \\
GenSD & Dim.3 & 0.33 & 0.0416849 \\
intermediate & Dim.3 & -0.35 & 0.0293668 \\
path & Dim.3 & -0.36 & 0.0276975 \\
top & Dim.3 & -0.38 & 0.0201240 \\
LinkSD & Dim.3 & -0.39 & 0.0148989 \\
VulSD & Dim.3 & -0.42 & 0.0094597 \\
ChSD & Dim.3 & -0.45 & 0.0047583 \\
ChNum & Dim.3 & -0.47 & 0.0026959 \\
ChLen & Dim.3 & -0.49 & 0.0018937 \\
TL & Dim.3 & -0.66 & 0.0000064 \\

\end{longtable}

}

\textsubscript{Source:
\href{https://BecksLab.github.io/ms_feature_selection/index.qmd.html}{Article
Notebook}}

\begin{figure}[H]

{\centering \pandocbounded{\includegraphics[keepaspectratio,alt={VERMAAT networks only}]{figures/pca_vermaat.png}}

}

\caption{VERMAAT networks only}

\end{figure}%

\begin{figure}[H]

{\centering \pandocbounded{\includegraphics[keepaspectratio,alt={All networks. Vermaat subset = using only the structural measures from Vermaat}]{figures/pca_allNetworks.png}}

}

\caption{All networks. Vermaat subset = using only the structural
measures from Vermaat}

\end{figure}%

\begin{figure}[H]

{\centering \pandocbounded{\includegraphics[keepaspectratio,alt={Cluster}]{figures/metric_hclust.png}}

}

\caption{Cluster}

\end{figure}%

\section*{References}\label{references}
\addcontentsline{toc}{section}{References}

\protect\phantomsection\label{refs}
\begin{CSLReferences}{1}{0}
\bibitem[\citeproctext]{ref-dunneFoodwebStructureNetwork2002}
Dunne, J. A., Williams, R. J., \& Martinez, N. D. (2002). Food-web
structure and network theory: {The} role of connectance and size.
\emph{Proceedings of the National Academy of Sciences}, \emph{99}(20),
12917--12922. \url{https://doi.org/10.1073/pnas.192407699}

\bibitem[\citeproctext]{ref-estradaUsingNetworkCentrality2008}
Estrada, E., \& Bodin, Ö. (2008). Using {Network Centrality Measures} to
{Manage Landscape Connectivity}. \emph{Ecological Applications},
\emph{18}(7), 1810--1825. \url{https://doi.org/10.1890/07-1419.1}

\bibitem[\citeproctext]{ref-jonssonReliabilityR50Measure2015}
Jonsson, T., Berg, S., Pimenov, A., Palmer, C., \& Emmerson, M. (2015).
The reliability of {R50} as a measure of vulnerability of food webs to
sequential species deletions. \emph{Oikos}, \emph{124}(4), 446--457.
\url{https://doi.org/10.1111/oik.01588}

\bibitem[\citeproctext]{ref-lauEcologicalNetworkMetrics2017}
Lau, M. K., Borrett, S. R., Baiser, B., Gotelli, N. J., \& Ellison, A.
M. (2017). Ecological network metrics: Opportunities for synthesis.
\emph{Ecosphere}, \emph{8}(8), e01900.
\url{https://doi.org/10.1002/ecs2.1900}

\bibitem[\citeproctext]{ref-mccannDiversityStabilityDebate2000}
McCann, K. S. (2000). The diversity--stability debate. \emph{Nature},
\emph{405}(6783), 228--233. \url{https://doi.org/10.1038/35012234}

\bibitem[\citeproctext]{ref-miloNetworkMotifsSimple2002}
Milo, R., Shen-Orr, S., Itzkovitz, S., Kashtan, N., Chklovskii, D., \&
Alon, U. (2002). Network {Motifs}: {Simple Building Blocks} of {Complex
Networks}. \emph{Science}, \emph{298}(5594), 824--827.
\url{https://doi.org/10.1126/science.298.5594.824}

\bibitem[\citeproctext]{ref-neutelStabilityRealFood2002}
Neutel, A.-M., Heesterbeek, J. A. P., \& De Ruiter, P. C. (2002).
Stability in {Real Food Webs}: {Weak Links} in {Long Loops}.
\emph{Science}, \emph{296}(5570), 1120--1123.
\url{https://doi.org/10.1126/science.1068326}

\bibitem[\citeproctext]{ref-stoufferCompartmentalizationIncreasesFoodweb2011}
Stouffer, D. B., \& Bascompte, J. (2011). Compartmentalization increases
food-web persistence. \emph{Proceedings of the National Academy of
Sciences of the United States of America}, \emph{108}(9), 3648--3652.
\url{https://doi.org/10.1073/pnas.1014353108}

\bibitem[\citeproctext]{ref-stoufferRobustMeasureFood2006a}
Stouffer, D. B., Camacho, J., \& Amaral, L. A. N. (2006). A robust
measure of food web intervality. \emph{Proceedings of the National
Academy of Sciences}, \emph{103}(50), 19015--19020.
\url{https://doi.org/10.1073/pnas.0603844103}

\bibitem[\citeproctext]{ref-stoufferEvidenceExistenceRobust2007}
Stouffer, D. B., Camacho, J., Jiang, W., \& Nunes Amaral, L. A. (2007).
Evidence for the existence of a robust pattern of prey selection in food
webs. \emph{Proceedings of the Royal Society B: Biological Sciences},
\emph{274}(1621), 1931--1940.
\url{https://doi.org/10.1098/rspb.2007.0571}

\bibitem[\citeproctext]{ref-strydomSVDEntropyReveals2021}
Strydom, T., Dalla Riva, G. V., \& Poisot, T. (2021). {SVD Entropy
Reveals} the {High Complexity} of {Ecological Networks}. \emph{Frontiers
in Ecology and Evolution}, \emph{9}.
\url{https://doi.org/10.3389/fevo.2021.623141}

\bibitem[\citeproctext]{ref-thompsonFoodWebsReconciling2012}
Thompson, R. M., Brose, U., Dunne, J., Hall, R. O., Hladyz, S.,
Kitching, R. L., Martinez, N. D., Rantala, H., Romanuk, T. N., Stouffer,
D. B., \& Tylianakis, J. M. (2012). Food webs: Reconciling the structure
and function of biodiversity. \emph{Trends in Ecology \& Evolution},
\emph{27}(12), 689--697.
\url{https://doi.org/10.1016/j.tree.2012.08.005}

\bibitem[\citeproctext]{ref-ulanowiczInformationTheoryEcology2001}
Ulanowicz, R. E. (2001). Information theory in ecology. \emph{Computers
\& Chemistry}, \emph{25}(4), 393--399.
\url{https://doi.org/10.1016/S0097-8485(01)00073-0}

\bibitem[\citeproctext]{ref-vermaatMajorDimensionsFoodweb2009}
Vermaat, J. E., Dunne, J. A., \& Gilbert, A. J. (2009). Major dimensions
in food-web structure properties. \emph{Ecology}, \emph{90}(1),
278--282. \url{https://doi.org/10.1890/07-0978.1}

\bibitem[\citeproctext]{ref-wattsCollectiveDynamicsSmallworld1998}
Watts, D. J., \& Strogatz, S. H. (1998). Collective dynamics of
{``small-world''} networks. \emph{Nature}, \emph{393}(6684), 440--442.
\url{https://doi.org/10.1038/30918}

\bibitem[\citeproctext]{ref-williamsLimitsTrophicLevels2004}
Williams, R. J., \& Martinez, N. D. (2004). Limits to {Trophic Levels}
and {Omnivory} in {Complex Food Webs}: {Theory} and {Data}. \emph{The
American Naturalist}, \emph{163}(3), 458--468.
\url{https://doi.org/10.1086/381964}

\bibitem[\citeproctext]{ref-yodzisSearchOperationalTrophospecies1999}
Yodzis, P., \& Winemiller, K. O. (1999). In {Search} of {Operational
Trophospecies} in a {Tropical Aquatic Food Web}. \emph{Oikos},
\emph{87}(2), 327--340. \url{https://doi.org/10.2307/3546748}

\end{CSLReferences}





\end{document}
