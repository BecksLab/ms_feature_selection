% Options for packages loaded elsewhere
% Options for packages loaded elsewhere
\PassOptionsToPackage{unicode}{hyperref}
\PassOptionsToPackage{hyphens}{url}
\PassOptionsToPackage{dvipsnames,svgnames,x11names}{xcolor}
%
\documentclass[
]{article}
\usepackage{xcolor}
\usepackage{amsmath,amssymb}
\setcounter{secnumdepth}{5}
\usepackage{iftex}
\ifPDFTeX
  \usepackage[T1]{fontenc}
  \usepackage[utf8]{inputenc}
  \usepackage{textcomp} % provide euro and other symbols
\else % if luatex or xetex
  \usepackage{unicode-math} % this also loads fontspec
  \defaultfontfeatures{Scale=MatchLowercase}
  \defaultfontfeatures[\rmfamily]{Ligatures=TeX,Scale=1}
\fi
\usepackage{lmodern}
\ifPDFTeX\else
  % xetex/luatex font selection
\fi
% Use upquote if available, for straight quotes in verbatim environments
\IfFileExists{upquote.sty}{\usepackage{upquote}}{}
\IfFileExists{microtype.sty}{% use microtype if available
  \usepackage[]{microtype}
  \UseMicrotypeSet[protrusion]{basicmath} % disable protrusion for tt fonts
}{}
\makeatletter
\@ifundefined{KOMAClassName}{% if non-KOMA class
  \IfFileExists{parskip.sty}{%
    \usepackage{parskip}
  }{% else
    \setlength{\parindent}{0pt}
    \setlength{\parskip}{6pt plus 2pt minus 1pt}}
}{% if KOMA class
  \KOMAoptions{parskip=half}}
\makeatother
% Make \paragraph and \subparagraph free-standing
\makeatletter
\ifx\paragraph\undefined\else
  \let\oldparagraph\paragraph
  \renewcommand{\paragraph}{
    \@ifstar
      \xxxParagraphStar
      \xxxParagraphNoStar
  }
  \newcommand{\xxxParagraphStar}[1]{\oldparagraph*{#1}\mbox{}}
  \newcommand{\xxxParagraphNoStar}[1]{\oldparagraph{#1}\mbox{}}
\fi
\ifx\subparagraph\undefined\else
  \let\oldsubparagraph\subparagraph
  \renewcommand{\subparagraph}{
    \@ifstar
      \xxxSubParagraphStar
      \xxxSubParagraphNoStar
  }
  \newcommand{\xxxSubParagraphStar}[1]{\oldsubparagraph*{#1}\mbox{}}
  \newcommand{\xxxSubParagraphNoStar}[1]{\oldsubparagraph{#1}\mbox{}}
\fi
\makeatother


\usepackage{longtable,booktabs,array}
\newcounter{none} % for unnumbered tables
\usepackage{calc} % for calculating minipage widths
% Correct order of tables after \paragraph or \subparagraph
\usepackage{etoolbox}
\makeatletter
\patchcmd\longtable{\par}{\if@noskipsec\mbox{}\fi\par}{}{}
\makeatother
% Allow footnotes in longtable head/foot
\IfFileExists{footnotehyper.sty}{\usepackage{footnotehyper}}{\usepackage{footnote}}
\makesavenoteenv{longtable}
\usepackage{graphicx}
\makeatletter
\newsavebox\pandoc@box
\newcommand*\pandocbounded[1]{% scales image to fit in text height/width
  \sbox\pandoc@box{#1}%
  \Gscale@div\@tempa{\textheight}{\dimexpr\ht\pandoc@box+\dp\pandoc@box\relax}%
  \Gscale@div\@tempb{\linewidth}{\wd\pandoc@box}%
  \ifdim\@tempb\p@<\@tempa\p@\let\@tempa\@tempb\fi% select the smaller of both
  \ifdim\@tempa\p@<\p@\scalebox{\@tempa}{\usebox\pandoc@box}%
  \else\usebox{\pandoc@box}%
  \fi%
}
% Set default figure placement to htbp
\def\fps@figure{htbp}
\makeatother


% definitions for citeproc citations
\NewDocumentCommand\citeproctext{}{}
\NewDocumentCommand\citeproc{mm}{%
  \begingroup\def\citeproctext{#2}\cite{#1}\endgroup}
\makeatletter
 % allow citations to break across lines
 \let\@cite@ofmt\@firstofone
 % avoid brackets around text for \cite:
 \def\@biblabel#1{}
 \def\@cite#1#2{{#1\if@tempswa , #2\fi}}
\makeatother
\newlength{\cslhangindent}
\setlength{\cslhangindent}{1.5em}
\newlength{\csllabelwidth}
\setlength{\csllabelwidth}{3em}
\newenvironment{CSLReferences}[2] % #1 hanging-indent, #2 entry-spacing
 {\begin{list}{}{%
  \setlength{\itemindent}{0pt}
  \setlength{\leftmargin}{0pt}
  \setlength{\parsep}{0pt}
  % turn on hanging indent if param 1 is 1
  \ifodd #1
   \setlength{\leftmargin}{\cslhangindent}
   \setlength{\itemindent}{-1\cslhangindent}
  \fi
  % set entry spacing
  \setlength{\itemsep}{#2\baselineskip}}}
 {\end{list}}
\usepackage{calc}
\newcommand{\CSLBlock}[1]{\hfill\break\parbox[t]{\linewidth}{\strut\ignorespaces#1\strut}}
\newcommand{\CSLLeftMargin}[1]{\parbox[t]{\csllabelwidth}{\strut#1\strut}}
\newcommand{\CSLRightInline}[1]{\parbox[t]{\linewidth - \csllabelwidth}{\strut#1\strut}}
\newcommand{\CSLIndent}[1]{\hspace{\cslhangindent}#1}



\setlength{\emergencystretch}{3em} % prevent overfull lines

\providecommand{\tightlist}{%
  \setlength{\itemsep}{0pt}\setlength{\parskip}{0pt}}



 


\makeatletter
\@ifpackageloaded{caption}{}{\usepackage{caption}}
\AtBeginDocument{%
\ifdefined\contentsname
  \renewcommand*\contentsname{Table of contents}
\else
  \newcommand\contentsname{Table of contents}
\fi
\ifdefined\listfigurename
  \renewcommand*\listfigurename{List of Figures}
\else
  \newcommand\listfigurename{List of Figures}
\fi
\ifdefined\listtablename
  \renewcommand*\listtablename{List of Tables}
\else
  \newcommand\listtablename{List of Tables}
\fi
\ifdefined\figurename
  \renewcommand*\figurename{Figure}
\else
  \newcommand\figurename{Figure}
\fi
\ifdefined\tablename
  \renewcommand*\tablename{Table}
\else
  \newcommand\tablename{Table}
\fi
}
\@ifpackageloaded{float}{}{\usepackage{float}}
\floatstyle{ruled}
\@ifundefined{c@chapter}{\newfloat{codelisting}{h}{lop}}{\newfloat{codelisting}{h}{lop}[chapter]}
\floatname{codelisting}{Listing}
\newcommand*\listoflistings{\listof{codelisting}{List of Listings}}
\makeatother
\makeatletter
\makeatother
\makeatletter
\@ifpackageloaded{caption}{}{\usepackage{caption}}
\@ifpackageloaded{subcaption}{}{\usepackage{subcaption}}
\makeatother
\usepackage{bookmark}
\IfFileExists{xurl.sty}{\usepackage{xurl}}{} % add URL line breaks if available
\urlstyle{same}
\hypersetup{
  pdftitle={Finding the major descriptors of species networks},
  pdfauthor={Tanya Strydom; Andrew P. Beckerman},
  pdfkeywords={food web, structure, dimensionality reduction},
  colorlinks=true,
  linkcolor={blue},
  filecolor={Maroon},
  citecolor={Blue},
  urlcolor={Blue},
  pdfcreator={LaTeX via pandoc}}



\title{Finding the major descriptors of species networks}
\author{Tanya Strydom %
%
\textsuperscript{%
%
1%
}%
; Andrew P. Beckerman %
%
\textsuperscript{%
%
1%
}%
}
\date{2026-01-29}

\usepackage{setspace}
\usepackage[left]{lineno}
\usepackage[letterpaper]{geometry}

\usepackage[nolists,noheads,markers]{endfloat}
\geometry{margin=2.5cm}

\begin{document}

\thispagestyle{empty}
{\bfseries\sffamily\Large Finding the major descriptors of species
networks}
\vfil
Tanya Strydom %
%
\textsuperscript{%
%
1%
}%
; Andrew P. Beckerman %
%
\textsuperscript{%
%
1%
}%

\vfil
{\small
\textbf{Abstract:} TODO
\vfil
\textbf{Keywords:} %
food web, structure, %
dimensionality reduction%
}
\clearpage
\setcounter{page}{1}
\doublespacing
\linenumbers


\section{Introduction}\label{introduction}

To bridge the gap between the original paper and your new objectives,
your introduction could follow this logical flow:

The Evolution of Ecological Network Theory

The Hook: Acknowledge the foundational shift from viewing biodiversity
as a simple ``species count'' to viewing it as a complex web of
interactions.

The Baseline: Summarize the core findings of the paper you're expanding
on---specifically, how the architecture (e.g., compartmentalization
vs.~nestedness) affects stability differently in mutualistic vs.~trophic
networks.

The Need for Dimensionality

The Gap: Argue that while ``connectance'' and ``nestedness'' are vital,
they don't capture the full resolution of ecosystem dynamics.

The Expansion: Introduce the necessity of more nuanced metrics (e.g.,
motifs, centrality, modularity, and beta-diversity of interactions) to
capture the ``hidden'' stability of diverse networks.

Linking Structure to Ecosystem Function (EF)

The Framework: Explicitly connect structural metrics to the
``Stability-Complexity'' debate.

The Hypothesis: Propose how specific structural arrangements (like high
modularity) act as ``firewalls'' to prevent the spread of perturbations,
thereby maintaining ecosystem function under stress.

Objectives

The overarching goal of this study is to move beyond bipartite
generalizations and define a comprehensive ``structural fingerprint'' of
ecosystem stability. To achieve this, we address two primary objectives:

\textbf{Identification of a Core Structural Subset}

Ecological networks are characterized by a high degree of collinearity
among structural descriptors. We aim to determine whether the 31 metrics
analyzed in this study can be reduced to a~\textbf{Minimum Sufficient
Set}---a small, non-redundant group of indicators that capture the
essential topological features of an ecosystem. By employing
multivariate techniques such as~\textbf{Variable
Clustering}~and~\textbf{SVD Complexity}, we seek to move away from
arbitrary metric selection toward a data-driven framework for network
characterization.

\textbf{Mapping the Multi-dimensional Stability Landscape}

Building on the ``stability-complexity'' debate (McCann 2000, Ives \&
Carpenter 2007), we aim to map how these diverse structural metrics
correlate with different facets of ecosystem health. Specifically, we
test the following hypotheses:

\begin{itemize}
\item
  \textbf{The Robustness Hypothesis:}~Metrics of redundancy
  (e.g.,~\emph{Connectance, MaxSim}) will be the primary predictors of
  resistance to primary species loss.
\item
  \textbf{The Containment Hypothesis:}~Modular structures
  (e.g.,~\emph{Clust, Modularity}) will correlate with system-wide
  resilience by preventing the propagation of local perturbations.
\item
  \textbf{The Dynamic Capacity Hypothesis:}~Information-theoretic
  measures (e.g.,~\emph{SVD Complexity, Spectral Radius}) will provide a
  superior bridge between static topology and the dynamic ability of the
  system to return to equilibrium.
\end{itemize}

Clearly state that this study expands the taxonomic and structural scope
of previous models to provide a generalized rulebook for
network-mediated stability.

Synthesis: Linking to ``Stability''

In your manuscript, you can group these metrics into~\textbf{three
functional categories}:

\begin{enumerate}
\def\labelenumi{\arabic{enumi}.}
\item
  \textbf{Robustness Metrics:}~(Richness, Connectance, Robustness,
  MaxSim) --- These describe how many ``hits'' the network can take
  before collapsing.
\item
  \textbf{Efficiency/Flow Metrics:}~(Path, ChLen, TL, Diameter) ---
  These describe how quickly energy or perturbations move through the
  system.
\item
  \textbf{Organization Metrics:}~(ρ, Complexity, Modularity/Clust,
  Intervality) --- These describe the ``logic'' of the arrangement,
  which dictates whether the system behaves predictably or chaotically.
\end{enumerate}

Blah blah blah Vermaat et al. (2009)

\emph{``It is incumbent on network ecologists to establish clearly the
independence and uniqueness of the descriptive metrics used.''} - Lau et
al. (2017)

\begin{longtable}[]{@{}
  >{\raggedright\arraybackslash}p{(\linewidth - 6\tabcolsep) * \real{0.2500}}
  >{\raggedright\arraybackslash}p{(\linewidth - 6\tabcolsep) * \real{0.2500}}
  >{\raggedright\arraybackslash}p{(\linewidth - 6\tabcolsep) * \real{0.2500}}
  >{\raggedright\arraybackslash}p{(\linewidth - 6\tabcolsep) * \real{0.2500}}@{}}
\caption{Stuff}\tabularnewline
\toprule\noalign{}
\begin{minipage}[b]{\linewidth}\raggedright
Dimension
\end{minipage} & \begin{minipage}[b]{\linewidth}\raggedright
Key Metrics
\end{minipage} & \begin{minipage}[b]{\linewidth}\raggedright
Expected Effect on Stability
\end{minipage} & \begin{minipage}[b]{\linewidth}\raggedright
Supporting Literature
\end{minipage} \\
\midrule\noalign{}
\endfirsthead
\toprule\noalign{}
\begin{minipage}[b]{\linewidth}\raggedright
Dimension
\end{minipage} & \begin{minipage}[b]{\linewidth}\raggedright
Key Metrics
\end{minipage} & \begin{minipage}[b]{\linewidth}\raggedright
Expected Effect on Stability
\end{minipage} & \begin{minipage}[b]{\linewidth}\raggedright
Supporting Literature
\end{minipage} \\
\midrule\noalign{}
\endhead
\bottomrule\noalign{}
\endlastfoot
Complexity \& Redundancy & Connectance, MaxSim, Links &
\textbf{Positive:} High redundancy allows for ``functional
compensation'' if one species is lost. & Dunne et al. (2002); McCann
(2000) \\
Compartmentalization & Clust, Modularity,~ρ & \textbf{Positive:} Limits
the spread of perturbations; local collapses don't become global. &
Stouffer \& Bascompte (2011) \\
Feedback \& Coupling & Omnivory (S2), Loop, ChLen & \textbf{Variable:}
Omnivory can stabilize by diffusing energy, but long chains can amplify
oscillations. & McCann (2000); Neutel et al. (2002) \\
Hierarchy \& Shape & Prey:Predator, Basal, Top & \textbf{Critical:}
``Bottom-heavy'' systems are generally more stable; inverted pyramids
are fragile. & \\
Information Heterogeneity & SVD Complexity, LinkSD & \textbf{Positive:}
Diverse interaction strengths prevent ``resonant'' instabilities. &
Ulanowicz (2001) \\
\end{longtable}

\section{Materials and Methods}\label{materials-and-methods}

\begin{longtable}[]{@{}
  >{\raggedright\arraybackslash}p{(\linewidth - 6\tabcolsep) * \real{0.0322}}
  >{\raggedright\arraybackslash}p{(\linewidth - 6\tabcolsep) * \real{0.3682}}
  >{\raggedright\arraybackslash}p{(\linewidth - 6\tabcolsep) * \real{0.4588}}
  >{\raggedright\arraybackslash}p{(\linewidth - 6\tabcolsep) * \real{0.1368}}@{}}
\caption{An informative caption about the different network properties.
We use a combination of metrics from both the original Vermaat et al.
(2009) paper as well as including those that have been identified by
Thompson et al. (2012) and have been linked to emerging ecosystem
properties such as stability}\label{tbl-properties}\tabularnewline
\toprule\noalign{}
\begin{minipage}[b]{\linewidth}\raggedright
Label
\end{minipage} & \begin{minipage}[b]{\linewidth}\raggedright
Definition
\end{minipage} & \begin{minipage}[b]{\linewidth}\raggedright
Ecological Significance
\end{minipage} & \begin{minipage}[b]{\linewidth}\raggedright
Reference (for maths), can make footnotes probs
\end{minipage} \\
\midrule\noalign{}
\endfirsthead
\toprule\noalign{}
\begin{minipage}[b]{\linewidth}\raggedright
Label
\end{minipage} & \begin{minipage}[b]{\linewidth}\raggedright
Definition
\end{minipage} & \begin{minipage}[b]{\linewidth}\raggedright
Ecological Significance
\end{minipage} & \begin{minipage}[b]{\linewidth}\raggedright
Reference (for maths), can make footnotes probs
\end{minipage} \\
\midrule\noalign{}
\endhead
\bottomrule\noalign{}
\endlastfoot
Basal & Percentage of basal taxa, defined as species who have a
vulnerability of zero & Measures the energy entry points; high basal \%
suggests a bottom-heavy, potentially more stable energy base. & \\
Connectance & \(L/S^2\), where \(S\) is the number of species and \(L\)
the number of links & & \\
Cannibal & Percentage of species that are cannibals & & \\
ChLen & Mean food chain length, averaged over all species (where a food
chain is defined as a continuous path from a `basal' to a `top' species)
& Reflects energy transfer efficiency. Longer chains may be more prone
to top-down trophic cascades. & \\
ChSD & Standard deviation of ChLen & High SD indicates a mix of energy
pathways, which can buffer the system & \\
ChNum & log number of food chains & & \\
Clust & mean clustering coefficient (probability that two taxa linked to
the same taxon are also linked) & Quantifies local redundancy; high
clustering can buffer the network against the loss of specific
interaction pathways. & \textbf{TODO}

Watts \& Strogatz (1998) \\
GenSD & Normalized standard deviation of generality of a species
standardized by \(L/S\) & Interaction asymmetry. High variance in how
links are distributed often points to the presence of `hubs' (highly
connected species), which makes the network robust to random loss but
vulnerable to targeted `keystone' removal. & Williams \& Martinez
(2008) \\
Herbivore & Percentage of herbivores plus detritivores (taxa that feed
only on basal taxa) & & \\
Intermediate & Percentage of intermediate taxa (with both consumers and
resources) & & \\
LinkSD & Normalized standard deviation of links (number of consumers
plus resources per taxon) & Interaction asymmetry. High variance in how
links are distributed often points to the presence of `hubs' (highly
connected species), which makes the network robust to random loss but
vulnerable to targeted `keystone' removal. & \\
Loop & Percentage of taxa in loops (food chains in which a taxon occurs
twice) & High percentages of loops can lead to feedback cycles (positive
or negative) that either amplify or dampen oscillations, directly
impacting local stability. & \\
L/S & links per species & & \\
MaxSim & Mean of the maximum trophic similarity of each taxon to other
taxa, the number of predators and prey shared by a pair of species
divided by their total number of predators and prey & Indicates
functional redundancy; high similarity suggests species are replaceable,
increasing robustness to individual extinctions. & \textbf{TODO}

Yodzis \& Winemiller (1999) \\
Omnivory & Percentage of omnivores (taxa that feed on \(\geq\) 2 taxa
with different trophic levels) & Links to coupling of energy channels;
historically debated, but often found to stabilize food webs by
diffusing top-down pressure. & McCann (2000) \\
Path & characteristic path length, the mean shortest food chain length
between species pairs & & \\
Richness & Number of nodes in the network & & \\
TL & Prey-weighted trophic level averaged across taxa & & Williams \&
Martinez (2004) \\
Top & Percentage of top taxa (taxa without consumers) & & \\
VulSD & Normalized standard deviation of vulnerability of a species
standardized by \(L/S\) & Interaction asymmetry. High variance in how
links are distributed often points to the presence of `hubs' (highly
connected species), which makes the network robust to random loss but
vulnerable to targeted `keystone' removal. & \\
Links & The number of links in the network & & \\
Diameter & Diameter can also be measured as the average of the distances
between each pair of nodes in the network & & Delmas et al. (2019) \\
\(\rho\) & Spectral radius is a a conceptual analog to nestedness. It is
defined as the absolute value of the largest real part of the
eigenvalues of the \emph{undirected} adjacency matrix & Acts as a proxy
for system-wide resilience; captures the speed at which a system returns
to equilibrium after a small pulse perturbation. & Staniczenko et al.
(2013) \\
Complexity & SVD complexity of a network, defined as the Pielou entropy
of its singular values & Captures structural heterogeneity;
distinguishes between a truly complex system and one that is merely
large or `random'. & Strydom et al. (2021) \\
Centrality & Centrality is a measure of how `influential' a species is,
under various definitions of `influence'. & Centrality can help in
quantifying the importance of species in a network & Estrada \& Bodin
(2008) \\
S1 & Number of linear chains & Building blocks of stability
(compartmentalisation, Stouffer and Bascompte?) & Stouffer et al. (2007)
Milo et al. (2002) \\
S2 & Number of omnivory motifs & Building blocks of stability
(compartmentalisation, Stouffer and Bascompte?) & Stouffer et al. (2007)
Milo et al. (2002) \\
S4 & Number of apparent competition motifs & Building blocks of
stability (compartmentalisation, Stouffer and Bascompte?) & Stouffer et
al. (2007) Milo et al. (2002) \\
S5 & Number of direct competition motifs & Building blocks of stability
(compartmentalisation, Stouffer and Bascompte?) & Stouffer et al. (2007)
Milo et al. (2002) \\
Intervality & The degree to which the prey in a food web can be ordered
so that all species can be placed along a single dimension & Measures
niche dimension; high intervality suggests a simpler organization where
species feeding habits are constrained by a single trait (like body
size). & Stouffer et al. (2006) \\
Prey:predator & Ratio of prey (basal + intermediate) to predators (top +
intermediate) & A measure of food web `shape'. Values \textless1 imply
an inverted structure and might indicate instability & \\
Robustness & Minimum level of secondary extinction that occurs in
response to a particular perturbation & & Jonsson et al. (2015) \\
\end{longtable}

{

\begin{longtable}[]{@{}llll@{}}

\caption{\label{tbl-corr}Here is a table showing the correlation of the
different network properties with the first three dimensions of the PCA}

\tabularnewline

\toprule\noalign{}
Property & PCA 1 (30\%) & PCA 2 (20\%) & PCA 3 (17\%) \\
\midrule\noalign{}
\endhead
\bottomrule\noalign{}
\endlastfoot
richness & 0.3 & \textbf{0.89} & -0.16 \\
links & 0.62 & \textbf{0.72} & 0.04 \\
connectance & 0.52 & -0.62 & 0.49 \\
diameter & \textbf{0.74} & 0.38 & -0.3 \\
complexity & -0.52 & 0.09 & -0.49 \\
distance & 0 & 0.3 & 0.18 \\
basal & -0.47 & 0.29 & \textbf{0.75} \\
top & -0.58 & 0.2 & -0.24 \\
intermediate & \textbf{0.69} & -0.35 & -0.52 \\
predpreyRatio & -0.26 & 0.27 & \textbf{0.76} \\
herbivory & -0.54 & 0.22 & 0.07 \\
omnivory & \textbf{0.78} & -0.23 & -0.21 \\
cannibal & \textbf{0.72} & 0.07 & 0.31 \\
l\_S & \textbf{0.83} & 0.47 & 0.23 \\
GenSD & -0.4 & 0.58 & 0.45 \\
VulSD & -0.41 & 0.58 & -0.26 \\
TL & 0.52 & -0.24 & \textbf{-0.77} \\
ChLen & 0.51 & -0.41 & -0.62 \\
ChSD & 0.32 & 0.2 & -0.45 \\
ChNum & -0.2 & \textbf{0.8} & -0.3 \\
path & 0.26 & 0.4 & -0.26 \\
LinkSD & -0.27 & \textbf{0.74} & -0.23 \\
S1 & \textbf{0.9} & 0.03 & 0.03 \\
S2 & \textbf{0.84} & -0.07 & 0.36 \\
S4 & 0.61 & 0.49 & 0.28 \\
S5 & \textbf{0.67} & 0.39 & 0.49 \\
ρ & 0.57 & -0.43 & 0.48 \\
centrality & -0.24 & \textbf{-0.67} & 0.18 \\
loops & \textbf{0.8} & 0.32 & 0.12 \\
robustness & 0.05 & -0.05 & 0.66 \\
intervals & 0.45 & \textbf{0.7} & -0.05 \\
MaxSim & -0.03 & -0.17 & 0.6 \\
Clust & \textbf{0.69} & -0.33 & 0.06 \\

\end{longtable}

}

\textsubscript{Source:
\href{https://BecksLab.github.io/ms_feature_selection/index.qmd.html}{Article
Notebook}}

\begin{figure}[H]

{\centering \pandocbounded{\includegraphics[keepaspectratio]{figures/pca_vermaat.png}}

}

\caption{VERMAAT networks only}

\end{figure}%

\begin{figure}[H]

{\centering \pandocbounded{\includegraphics[keepaspectratio]{figures/pca_allNetworks.png}}

}

\caption{All networks. Vermaat subset = using only the structural
measures from Vermaat}

\end{figure}%

\section*{References}\label{references}
\addcontentsline{toc}{section}{References}

\protect\phantomsection\label{refs}
\begin{CSLReferences}{1}{0}
\bibitem[\citeproctext]{ref-delmasAnalysingEcologicalNetworks2019}
Delmas, E., Besson, M., Brice, M.-H., Burkle, L. A., Riva, G. V. D.,
Fortin, M.-J., Gravel, D., Guimarães, P. R., Hembry, D. H., Newman, E.
A., Olesen, J. M., Pires, M. M., Yeakel, J. D., \& Poisot, T. (2019).
Analysing ecological networks of species interactions. \emph{Biological
Reviews}, \emph{94}(1), 16--36. \url{https://doi.org/10.1111/brv.12433}

\bibitem[\citeproctext]{ref-dunneFoodwebStructureNetwork2002}
Dunne, J. A., Williams, R. J., \& Martinez, N. D. (2002). Food-web
structure and network theory: {The} role of connectance and size.
\emph{Proceedings of the National Academy of Sciences}, \emph{99}(20),
12917--12922. \url{https://doi.org/10.1073/pnas.192407699}

\bibitem[\citeproctext]{ref-estradaUsingNetworkCentrality2008}
Estrada, E., \& Bodin, Ö. (2008). Using {Network Centrality Measures} to
{Manage Landscape Connectivity}. \emph{Ecological Applications},
\emph{18}(7), 1810--1825. \url{https://doi.org/10.1890/07-1419.1}

\bibitem[\citeproctext]{ref-jonssonReliabilityR50Measure2015}
Jonsson, T., Berg, S., Pimenov, A., Palmer, C., \& Emmerson, M. (2015).
The reliability of {R50} as a measure of vulnerability of food webs to
sequential species deletions. \emph{Oikos}, \emph{124}(4), 446--457.
\url{https://doi.org/10.1111/oik.01588}

\bibitem[\citeproctext]{ref-lauEcologicalNetworkMetrics2017}
Lau, M. K., Borrett, S. R., Baiser, B., Gotelli, N. J., \& Ellison, A.
M. (2017). Ecological network metrics: Opportunities for synthesis.
\emph{Ecosphere}, \emph{8}(8), e01900.
\url{https://doi.org/10.1002/ecs2.1900}

\bibitem[\citeproctext]{ref-mccannDiversityStabilityDebate2000}
McCann, K. S. (2000). The diversity--stability debate. \emph{Nature},
\emph{405}(6783), 228--233. \url{https://doi.org/10.1038/35012234}

\bibitem[\citeproctext]{ref-miloNetworkMotifsSimple2002}
Milo, R., Shen-Orr, S., Itzkovitz, S., Kashtan, N., Chklovskii, D., \&
Alon, U. (2002). Network {Motifs}: {Simple Building Blocks} of {Complex
Networks}. \emph{Science}, \emph{298}(5594), 824--827.
\url{https://doi.org/10.1126/science.298.5594.824}

\bibitem[\citeproctext]{ref-neutelStabilityRealFood2002}
Neutel, A.-M., Heesterbeek, J. A. P., \& De Ruiter, P. C. (2002).
Stability in {Real Food Webs}: {Weak Links} in {Long Loops}.
\emph{Science}, \emph{296}(5570), 1120--1123.
\url{https://doi.org/10.1126/science.1068326}

\bibitem[\citeproctext]{ref-staniczenkoGhostNestednessEcological2013}
Staniczenko, P. P. A., Kopp, J. C., \& Allesina, S. (2013). The ghost of
nestedness in ecological networks. \emph{Nature Communications},
\emph{4}(1), 1391. \url{https://doi.org/10.1038/ncomms2422}

\bibitem[\citeproctext]{ref-stoufferCompartmentalizationIncreasesFoodweb2011}
Stouffer, D. B., \& Bascompte, J. (2011). Compartmentalization increases
food-web persistence. \emph{Proceedings of the National Academy of
Sciences of the United States of America}, \emph{108}(9), 3648--3652.
\url{https://doi.org/10.1073/pnas.1014353108}

\bibitem[\citeproctext]{ref-stoufferRobustMeasureFood2006a}
Stouffer, D. B., Camacho, J., \& Amaral, L. A. N. (2006). A robust
measure of food web intervality. \emph{Proceedings of the National
Academy of Sciences}, \emph{103}(50), 19015--19020.
\url{https://doi.org/10.1073/pnas.0603844103}

\bibitem[\citeproctext]{ref-stoufferEvidenceExistenceRobust2007}
Stouffer, D. B., Camacho, J., Jiang, W., \& Nunes Amaral, L. A. (2007).
Evidence for the existence of a robust pattern of prey selection in food
webs. \emph{Proceedings of the Royal Society B: Biological Sciences},
\emph{274}(1621), 1931--1940.
\url{https://doi.org/10.1098/rspb.2007.0571}

\bibitem[\citeproctext]{ref-strydomSVDEntropyReveals2021}
Strydom, T., Dalla Riva, G. V., \& Poisot, T. (2021). {SVD Entropy
Reveals} the {High Complexity} of {Ecological Networks}. \emph{Frontiers
in Ecology and Evolution}, \emph{9}.
\url{https://doi.org/10.3389/fevo.2021.623141}

\bibitem[\citeproctext]{ref-thompsonFoodWebsReconciling2012}
Thompson, R. M., Brose, U., Dunne, J., Hall, R. O., Hladyz, S.,
Kitching, R. L., Martinez, N. D., Rantala, H., Romanuk, T. N., Stouffer,
D. B., \& Tylianakis, J. M. (2012). Food webs: Reconciling the structure
and function of biodiversity. \emph{Trends in Ecology \& Evolution},
\emph{27}(12), 689--697.
\url{https://doi.org/10.1016/j.tree.2012.08.005}

\bibitem[\citeproctext]{ref-ulanowiczInformationTheoryEcology2001}
Ulanowicz, R. E. (2001). Information theory in ecology. \emph{Computers
\& Chemistry}, \emph{25}(4), 393--399.
\url{https://doi.org/10.1016/S0097-8485(01)00073-0}

\bibitem[\citeproctext]{ref-vermaatMajorDimensionsFoodweb2009}
Vermaat, J. E., Dunne, J. A., \& Gilbert, A. J. (2009). Major dimensions
in food-web structure properties. \emph{Ecology}, \emph{90}(1),
278--282. \url{https://doi.org/10.1890/07-0978.1}

\bibitem[\citeproctext]{ref-wattsCollectiveDynamicsSmallworld1998}
Watts, D. J., \& Strogatz, S. H. (1998). Collective dynamics of
{``small-world''} networks. \emph{Nature}, \emph{393}(6684), 440--442.
\url{https://doi.org/10.1038/30918}

\bibitem[\citeproctext]{ref-williamsLimitsTrophicLevels2004}
Williams, R. J., \& Martinez, N. D. (2004). Limits to {Trophic Levels}
and {Omnivory} in {Complex Food Webs}: {Theory} and {Data}. \emph{The
American Naturalist}, \emph{163}(3), 458--468.
\url{https://doi.org/10.1086/381964}

\bibitem[\citeproctext]{ref-williamsSuccessItsLimits2008a}
Williams, R. J., \& Martinez, N. D. (2008). Success and its limits among
structural models of complex food webs. \emph{The Journal of Animal
Ecology}, \emph{77}(3), 512--519.
\url{https://doi.org/10.1111/j.1365-2656.2008.01362.x}

\bibitem[\citeproctext]{ref-yodzisSearchOperationalTrophospecies1999}
Yodzis, P., \& Winemiller, K. O. (1999). In {Search} of {Operational
Trophospecies} in a {Tropical Aquatic Food Web}. \emph{Oikos},
\emph{87}(2), 327--340. \url{https://doi.org/10.2307/3546748}

\end{CSLReferences}





\end{document}
